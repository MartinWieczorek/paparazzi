\section{Rates}
\subsection{Definition}
The values are called
\begin{equation}
\ra{} = \begin{pmatrix} p \\q\\r\end{pmatrix}
\end{equation}
It is available for the following simple types:\\
\begin{tabular}{c|c}
type		& struct		\\ \hline
int16\_t	& Int16Rates	\\
int32\_t	& Int32Rates	\\
float 		& FloatRates	\\
double 		& DoubleRates
\end{tabular}

\subsection{= Assigning}
\subsubsection*{$\ra{} = 0$}
\begin{equation}
\ra{} = \begin{pmatrix}p\\q\\r\end{pmatrix} = \begin{pmatrix}0\\0\\0\end{pmatrix}
\end{equation}
\inHfile{FLOAT\_RATES\_ZERO(r)}{pprz\_algebra\_float}

\subsubsection*{$\ra{} = \transp{(p, q, r)}$}
\begin{equation}
\ra = \transp{(p, q, r)}
\end{equation}
\inHfile{RATES\_ASSIGN(ra, p, q, r)}{pprz\_algebra}

\subsubsection*{$\ra a = \ra b$}
\begin{equation}
\ra a = \ra b
\end{equation}
\inHfile{RATES\_COPY(a, b)}{pprz\_algebra}



\subsection{+ Addition}
\subsubsection*{$\ra a += \ra b$}
\begin{equation}
\ra a = \ra a + \ra b
\end{equation}
\inHfile{RATES\_ADD(a, b)}{pprz\_algebra}

\subsubsection*{$\ra c = \ra a + \ra b$}
\begin{equation}
\ra c = \ra a + \ra b
\end{equation}
\inHfile{RATES\_SUM(c, a, b)}{pprz\_algebra}



\subsection{- Subtraction}
\subsubsection*{$\ra a -= \ra b$}
\begin{equation}
\ra a = \ra a - \ra b
\end{equation}
\inHfile{RATES\_SUB(a, b)}{pprz\_algebra}

\subsubsection*{$\ra c = \ra a - \ra b$}
\begin{equation}
\ra c = \ra a - \ra b
\end{equation}
\inHfile{RATES\_DIFF(c, a, b)}{pprz\_algebra}



\subsection{$\multiplication$ Multiplication}
\subsubsection*{$\ra{ro} = s \multiplication \ra{ri}$ With a scalar}
\begin{equation}
\ra{ro} = s \multiplication \ra{ri}
\end{equation}
\inHfile{RATES\_SMUL(ro, ri, s)}{pprz\_algebra}

\subsubsection*{$\ra c = 2^{-s} \multiplication \ra a \ew \multiplication \ra b$ Element-wise with bit-shift}
Makes an element-wise multiplication (also known as ``Dot-Multiplication'' from languages like MATLAB, FreeMat or Octave) and an additional bitshift to the right about s. The bitwise shift operation often results in a multiplication like $2^{-s}$, but especially for the divisions of integer values it's not the same.
\begin{equation}
\ra c = \begin{pmatrix}p_c\\q_c\\r_c\end{pmatrix} = 
\begin{pmatrix}
2^{-s} \; p_a \multiplication p_b \\
2^{-s} \; q_a \multiplication q_b \\
2^{-s} \; r_a \multiplication r_b
\end{pmatrix}
\end{equation}

\subsubsection*{$ \ra {vb} = \transp{\mat M_{b2a}} \multiplication \ra {va}$ With a rotational matrix}
\begin{equation}
\ra {vb} = \transp{\mat M_{b2a}} \multiplication \ra {va}
\end{equation}
\inHfile{INT32\_RMAT\_TRANSP\_RATEMULT(vb, m\_b2a, va)}{pprz\_algebra\_int}
\inHfile{FLOAT\_RMAT\_TRANSP\_RATEMULT(vb, m\_b2a, va)}{pprz\_algebra\_float}
or without the not-transposed matrix
\begin{equation}
\ra {vb} = \mat M_{a2b} \multiplication \ra {va}
\end{equation}
\inHfile{FLOAT\_RMAT\_RATEMULT(vb, m\_a2b, va)}{pprz\_algebra\_float}



\subsection{$\division$ Division}
\subsubsection*{$\ra{ro} = \frac 1 s \multiplication \ra{ri}$ With a scalar}
\begin{equation}
\ra{ro} = \frac 1 s \multiplication \ra{ri}
\end{equation}
\inHfile{EULERS\_SDIV(ro, ri, s)}{pprz\_algebra}


\subsection{Transformation form rates}
\subsubsection*{to euler angles (derivative)}
The transformation from euler angles derivative to rates can be written as a matrix multiplication
\begin{equation}
\begin{pmatrix} p\\q\\r \end{pmatrix} = 
\begin{pmatrix}
- \sin (\Roll) \dot{\Yaw} + \dot{\Roll} \\
\sin (\Roll) \cos (\Pitch) \dot{\Yaw} + \cos (\Roll) \dot{\Pitch} \\
\cos (\Roll) \cos (\Pitch) \dot{\Yaw} - \sin (\Roll) \dot{\Pitch}
\end{pmatrix} \Leftrightarrow \begin{pmatrix} p\\q\\r \end{pmatrix} = 
\begin{pmatrix}
1 & 0 				& -\sin (\Roll) \\
0 & \cos (\Roll)	& \sin (\Roll) \cos (\Pitch) \\
0 & -\sin (\Roll)	& \cos (\Roll) \cos (\Pitch)
\end{pmatrix} \multiplication \begin{pmatrix}
\dot{\Roll} \\
\dot{\Pitch} \\
\dot{\Yaw}
\end{pmatrix}.
\end{equation}
This can be solved easily to
\begin{equation}
\begin{pmatrix}\dot{\Roll} \\ \dot{\Pitch} \\ \dot{\Yaw} \end{pmatrix} = 
\begin{pmatrix}
1 & \frac{ \sin^2 \Roll }{\cos \Pitch}	& \frac{\sin \Roll \cos \Roll}{\cos \Pitch}	\\
0 & \cos \Roll							& -\sin \Roll	\\
0 & \frac{\sin \Roll}{\cos \Pitch}		& \frac{\cos \Roll}{\cos \Pitch}
\end{pmatrix} \multiplication \begin{pmatrix} p\\q\\r \end{pmatrix}.
\end{equation}
Please note the singularity at the \emph{gimbal lock} ($\Pitch = \pm 90^{\circ }$)!
\inHfile{INT32\_EULERS\_DOT\_OF\_RATES(ed, e, r)}{pprz\_algebra\_int}
\inHfile{INT32\_EULERS\_DOT\_321\_OF\_RATES(ed, e, r)}{pprz\_algebra\_int}


\subsubsection*{to a quaternion}
This function computes the differential quaternion of measured rates after an amount of time. \\
Let $ \ra{} $ be the measured rates, then $\norm {\ra {}}$ represents the absolute value of rates. Therefore,
\begin{equation}
\Delta \alpha = \norm {\ra {}} \multiplication \Delta t
\end{equation}
is the rotational angle. The (normalized) axis of the rotation is then
\begin{equation}
\vect v = \frac{\ra{}}{\norm {\ra {}}}.
\end{equation}
The construction of a quaternion from an axis and an angle is
\begin{equation}
\quat{} = \begin{pmatrix}
\cos \tfrac \alpha 2 \\
\vect v \sin \tfrac \alpha 2
\end{pmatrix},
\end{equation}
so that the resulting quaternion of measured rates becomes
\begin{equation}
\quat{} = \begin{pmatrix}
\cos \tfrac{\norm {\ra {}} \multiplication \Delta t} 2 \\
 \frac{\ra{}}{\norm {\ra {}}} \sin \tfrac{\norm {\ra {}} \multiplication \Delta t} 2
\end{pmatrix}.
\end{equation}
\inHfile{FLOAT\_QUAT\_DIFFERENTIAL(q\_out, w, dt)}{pprz\_algebra\_float}


\subsection{Transformation to rates}
\subsubsection*{from euler angles (derivative)}
This function requires the euler angles e and also their derivative ed.\\
\begin{equation}
\ra{r} = \begin{pmatrix} p\\q\\r \end{pmatrix} = 
\eye \multiplication \eye \multiplication\begin{pmatrix} \dot{\Roll}\\0\\0 \end{pmatrix} + 
\mat R(\Roll)\multiplication \eye \multiplication\begin{pmatrix} 0\\\dot{\Pitch}\\0 \end{pmatrix} + 
\mat R(\Roll)\multiplication \mat R(\Pitch)\multiplication \begin{pmatrix} 0\\0\\\dot{\Yaw} \end{pmatrix}
\end{equation}
\begin{equation}
\mat R(\Roll) = \begin{pmatrix}
1 & 0         &     0      \\
0 & cos(\Roll) & -sin(\Roll) \\
0 & sin(\Roll) &  cos(\Roll) 
\end{pmatrix}
\end{equation}
\begin{equation}
\mat R(\Pitch) = \begin{pmatrix}
 cos(\Pitch) & 0 & sin(\Pitch) \\
 0         & 1 &     0     \\
-sin(\Pitch) & 0 & cos(\Pitch) 
\end{pmatrix}
\end{equation}
\begin{equation}
\ra r = \begin{pmatrix} p\\q\\r \end{pmatrix} = 
\begin{pmatrix}
- \sin (\Roll) \dot{\Yaw} + \dot{\Roll} \\
\sin (\Roll) \cos (\Pitch) \dot{\Yaw} + \cos (\Roll) \dot{\Pitch} \\
\cos (\Roll) \cos (\Pitch) \dot{\Yaw} - \sin (\Roll) \dot{\Pitch} \\
\end{pmatrix}
\end{equation}
\inHfile{INT32\_RATES\_OF\_EULERS\_DOT(r, e, ed)}{pprz\_algebra\_int}
\inHfile{INT32\_RATES\_OF\_EULERS\_DOT\_321(r, e, ed)}{pprz\_algebra\_int}




\subsection{Other}
\subsubsection*{$\norm{\ra{}}$ Norm}
Computes the 2-norm of the rates (the length).
\begin{equation}
n = \norm{\ra{}} = \sqrt{p^2 +q^2+r^2}
\end{equation}
\inHfile{FLOAT\_RATES\_NORM(v)}{pprz\_algebra\_float}

\subsubsection*{$min \leq \ra v \leq max$ Bounding}
Bounds the rates so that every value (p, q, r) is between \textit{min} and \textit{max}.
\begin{equation}
\ra v \in \mathbb{I}^3, \quad \mathbb{I} = [min; max]
\end{equation}
The lower border \textit{min} has a higher priority than \textit{max}. So, if $ min > max$ and a value of $ \ra v $ is between those, the value is set to \textit{min}. \\
\inHfile{RATES\_BOUND\_CUBE(v, min, max)}{pprz\_algebra}
\mynote{See the note for euler angles and the naming is bad.}

\subsubsection*{$\ra{v_{min}} \leq \ra v \leq \ra{v_{max}}$ Bounding}
Ensures that
\begin{equation}
\ra {v_{min}} \leq \ra v \leq \ra{v_{max}} \Leftrightarrow \begin{pmatrix} p_{min} \\ q_{min} \\ r_{min} \end{pmatrix} \leq \begin{pmatrix} p \\ q \\ r \end{pmatrix} \leq \begin{pmatrix} p_{max} \\ q_{max} \\ r_{max} \end{pmatrix}
\end{equation}
The upper border \textit{max} has a higher priority than \textit{min}. So, if $ min > max$ and a value of $ \ra v $ is between those, the value is set to \textit{max}. \\
\inHfile{RATES\_BOUND\_BOX(v, v\_min, v\_max}{pprz\_algebra}
\mynote{Not very consequent with the priority}