\section{Euler Angles}
\subsection{Definition}
The values are called
\begin{equation}
\eu{e} = \begin{pmatrix} \Roll \\\Pitch\\\Yaw\\\end{pmatrix} = \begin{pmatrix} phi \\Pitch\\Yaw\\\end{pmatrix}
\end{equation}
It is available for the following simple types:\\
\begin{tabular}{c|c}
type		& struct		\\ \hline
int16\_t	& Int16Eulers	\\
int32\_t	& Int32Eulers	\\
float		& FloatEulers	\\
double		& DoubleEulers
\end{tabular}
\textbf{IMPORTANT:}\label{paparazzi euler definition}\\
Because there are many definitions of euler angles (some say 12, wikipedia says 24, the author tends to believe there are 48) and the choice of perspective, paparazzi choosed the following convention:




\subsection{= Assigning}
\subsubsection*{$\eu{e} = \eu{0}$}
\begin{equation}
\eu v = \begin{pmatrix} 0 \\ 0 \\ 0 \end{pmatrix}
\end{equation}
\inHfile{INT\_EULERS\_ZERO(e)}{pprz\_algebra\_int}
\inHfile{FLOAT\_EULERS\_ZERO(e)}{pprz\_algebra\_float}

\subsubsection*{$\eu a = \transp{(\Roll,\Pitch,\Yaw)}$}
\begin{equation}
\eu a = \transp{(\Roll,\Pitch,\Yaw)}
\end{equation}
\inHfile{EULERS\_ASSIGN(e, phi, theta, psi)}{pprz\_algebra}

\subsubsection*{$\eu a = \eu b$}
\begin{equation}
\eu a = \eu b
\end{equation}
\inHfile{EULERS\_COPY(a, b)}{pprz\_algebra}



\subsection{+ Addition}
\subsubsection*{$\eu a += \eu b$}
\begin{equation}
\eu a = \eu a + \eu b
\end{equation}
\inHfile{EULERS\_ADD(a, b)}{pprz\_algebra}
\mynote{No EULERS\_SUM function?}



\subsection{- Subtraction}
\subsubsection*{$\eu a -= \eu b$}
\begin{equation}
\eu a = \eu a - \eu b
\end{equation}
\inHfile{EULERS\_SUB(a, b)}{pprz\_algebra}

\subsubsection*{$\eu c = \eu a - \eu b$}
\begin{equation}
\eu c = \eu a - \eu b
\end{equation}
\inHfile{EULERS\_DIFF(c, a, b)}{pprz\_algebra}




\subsection{$\multiplication$ Multiplication}
\subsubsection*{$\eu{e_o} = s \multiplication \eu{e_i}$ With a scalar}
\begin{equation}
\eu e_o = s \multiplication \eu{e_i}
\end{equation}
\inHfile{EULERS\_SMUL(eo, ei, s)}{pprz\_algebra}




\subsection{$\division$ Division}
\subsubsection*{$\eu{e_o} = \frac 1 s \multiplication \eu{e_i}$ With a scalar}
\begin{equation}
\eu{e_o} = \frac 1 s \multiplication \eu{e_i}
\end{equation}
\inHfile{EULERS\_SDIV(eo, ei, s)}{pprz\_algebra}



\subsection{Transformation from euler angles}
\subsubsection*{to a rotational matrix}
The transformation from euler angles $ \eu e$ to a rotational matrix depends on the order of rotation. Here, the default order is 321, which means first \Yawc{Yaw} (about the \emph{third} axis), then \Pitchc{Pitch} (the \emph{second} axis) and finally \Rollc{Roll}(the \emph{first} axis). Please note the important definition about perspectives (page \ref{Important definition}).
\begin{equation}
\mat R_m = \begin{pmatrix}
cos(\Pitch)cos(\Yaw)									& cos(\Pitch)sin(\Yaw)									& -sin(\Pitch)			\\
sin(\Roll)sin(\Pitch)cos(\Yaw) - cos(\Roll)cos(\Yaw)	& sin(\Roll)sin(\Pitch)sin(\Yaw) + cos(\Roll)cos(\Yaw)	& sin(\Roll)cos(\Pitch)	\\
cos(\Roll)sin(\Pitch)cos(\Yaw) + sin(\Roll)sin(\Yaw)	& cos(\Roll)sin(\Pitch)sin(\Yaw) - sin(\Roll)cos(\Yaw)	& cos(\Roll)cos(\Pitch)
\end{pmatrix}\end{equation}

\inHfile{INT32\_RMAT\_OF\_EULERS(rm, e)}{pprz\_algebra\_int}
\inHfile{INT32\_RMAT\_OF\_EULERS\_321(rm, e)}{pprz\_algebra\_int}
\inHfile{FLOAT\_RMAT\_OF\_EULERS(rm, e)}{pprz\_algebra\_float}
\inHfile{FLOAT\_RMAT\_OF\_EULERS\_321(rm, e)}{pprz\_algebra\_float}
You can also choose the 312 definition (First \Yawc{Yaw}, then  \Rollc{Roll} then \Pitchc{Pitch} $\Rightarrow \mat R(\Yaw) \mat R(\Roll)  \mat R(\Pitch)$). Again, remember the different order and sign:
\begin{equation}
\mat R_m = \mat R(-\Pitch) \mat R(-\Roll)  \mat R(-\Yaw)
\end{equation}
\begin{equation}
\mat R_m = \begin{pmatrix}
cos(\Pitch)cos(\Yaw)-sin(\Roll)sin(\Pitch)sin(\Yaw)		& cos(\Pitch)sin(\Yaw) + sin(\Roll)sin(\Pitch)cos(\Yaw)	& -cos(\Roll)sin(\Pitch) \\
-cos(\Roll)sin(\Yaw)									& cos(\Roll)cos(\Yaw)									& sin(\Roll)		\\
sin(\Pitch)cos(\Yaw) + sin(\Roll)cos(\Pitch)sin(\Yaw)	& sin(\Pitch)sin(\Yaw)-sin(\Roll)cos(\Pitch)cos(\Yaw)	& cos(\Roll)cos(\Pitch)
\end{pmatrix}\end{equation}
\inHfile{INT32\_RMAT\_OF\_EULERS\_312(rm, e)}{pprz\_algebra\_int}
\inHfile{FLOAT\_RMAT\_OF\_EULERS\_312(rm, e)}{pprz\_algebra\_float}
\inHfile{DOUBLE\_RMAT\_OF\_EULERS\_312(rm, e)}{pprz\_algebra\_float}


\subsubsection*{to a quaternion}
The transformation is given by
\begin{equation}
\quat{} = [\cos \tfrac{\Yaw}{2} + \mathbf{k} \sin \tfrac{\Yaw}{2}][\cos \tfrac{\Pitch}{2} + \mathbf{j} \sin \tfrac{\Pitch}{2}][\cos \tfrac{\Roll}{2} + \mathbf{i} \sin \tfrac{\Roll}{2}]
\end{equation}
In matrix notation:
\begin{equation}
\quat = \begin{pmatrix}
\cos \tfrac{\Roll}{2} \cos \tfrac{\Pitch}{2} \cos \tfrac{\Yaw}{2} + \sin \tfrac{\Roll}{2} \sin \tfrac{\Pitch}{2} \sin \tfrac{\Yaw}{2} \\
\sin \tfrac{\Roll}{2} \cos \tfrac{\Pitch}{2} \cos \tfrac{\Yaw}{2} - \cos \tfrac{\Roll}{2} \sin \tfrac{\Pitch}{2} \sin \tfrac{\Yaw}{2} \\
\cos \tfrac{\Roll}{2} \sin \tfrac{\Pitch}{2} \cos \tfrac{\Yaw}{2} + \sin \tfrac{\Roll}{2} \cos \tfrac{\Pitch}{2} \sin \tfrac{\Yaw}{2} \\
\cos \tfrac{\Roll}{2} \cos \tfrac{\Pitch}{2} \sin \tfrac{\Yaw}{2} - \sin \tfrac{\Roll}{2} \cos \tfrac{\Pitch}{2} \sin \tfrac{\Yaw}{2}
\end{pmatrix}
\end{equation}
\inHfile{INT32\_QUAT\_OF\_EULERS(q, e)}{pprz\_algebra\_int}
\inHfile{FLOAT\_QUAT\_OF\_EULERS(q, e)}{pprz\_algebra\_float}
\inHfile{DOUBLE\_QUAT\_OF\_EULERS(q, e)}{pprz\_algebra\_double}


\subsubsection*{to rates}
This function requires the euler angles e and also their derivative ed.\\
\begin{equation}
\ra{r} = \begin{pmatrix} p\\q\\r \end{pmatrix} = 
\eye \multiplication \eye \multiplication\begin{pmatrix} \dot{\Roll}\\0\\0 \end{pmatrix} + 
\mat R(\Roll)\multiplication \eye \multiplication\begin{pmatrix} 0\\\dot{\Pitch}\\0 \end{pmatrix} + 
\mat R(\Roll)\multiplication \mat R(\Pitch)\multiplication \begin{pmatrix} 0\\0\\\dot{\Yaw} \end{pmatrix}
\end{equation}
\begin{equation}
\mat R(\Roll) = \begin{pmatrix}
1 & 0         &     0      \\
0 & cos(\Roll) & -sin(\Roll) \\
0 & sin(\Roll) &  cos(\Roll) 
\end{pmatrix}
\end{equation}
\begin{equation}
\mat R(\Pitch) = \begin{pmatrix}
 cos(\Pitch) & 0 & sin(\Pitch) \\
 0         & 1 &     0     \\
-sin(\Pitch) & 0 & cos(\Pitch) 
\end{pmatrix}
\end{equation}
\begin{equation}
\ra r = \begin{pmatrix} p\\q\\r \end{pmatrix} = 
\begin{pmatrix}
- \sin (\Roll) \dot{\Yaw} + \dot{\Roll} \\
\sin (\Roll) \cos (\Pitch) \dot{\Yaw} + \cos (\Roll) \dot{\Pitch} \\
\cos (\Roll) \cos (\Pitch) \dot{\Yaw} - \sin (\Roll) \dot{\Pitch} \\
\end{pmatrix}
\end{equation}
\inHfile{INT32\_RATES\_OF\_EULERS\_DOT(r, e, ed)}{pprz\_algebra\_int}
\inHfile{INT32\_RATES\_OF\_EULERS\_DOT\_321(r, e, ed)}{pprz\_algebra\_int}





\subsection{Transformation to euler angles}
\subsubsection*{form a rotational matrix}
\mynote{This is only for the 321-convention}
The rotation matrix from euler angles is known
\begin{equation}
\mat R_m = \begin{pmatrix}
cos(\Pitch)cos(\Yaw)									& cos(\Pitch)sin(\Yaw)									& -sin(\Pitch)			\\
sin(\Roll)sin(\Pitch)cos(\Yaw) - cos(\Roll)cos(\Yaw)	& sin(\Roll)sin(\Pitch)sin(\Yaw) + cos(\Roll)cos(\Yaw)	& sin(\Roll)cos(\Pitch)	\\
cos(\Roll)sin(\Pitch)cos(\Yaw) + sin(\Roll)sin(\Yaw)	& cos(\Roll)sin(\Pitch)sin(\Yaw) - sin(\Roll)cos(\Yaw)	& cos(\Roll)cos(\Pitch)
\end{pmatrix}
\end{equation}
and the extraction is done vice versa.
\begin{equation}
\eu e = \begin{pmatrix}\Roll \\ \Pitch \\ \Yaw \end{pmatrix} = 
\begin{pmatrix}
\arctan2(r_{23}, r_{33}) \\
-\arcsin(r_{13}) \\
\arctan2(r_{12}, r_{11})
\end{pmatrix}
\end{equation}
\inHfile{INT32\_EULERS\_OF\_RMAT(e, rm)}{pprz\_algebra\_int}
\inHfile{FLOAT\_EULERS\_OF\_RMAT(e, rm)}{pprz\_algebra\_float}


\subsubsection*{from a quaternion}
This is done by constructing a rotational matrix out of a quaternion (note: not all elements need to be generated), 
\begin{equation}
\mat R_m = \begin{pmatrix}
1-2(q_y^2 + q_z^2)		& 2(q_xq_y-q_iq_z)		& 2(q_xq_z + q_iq_y) \\
						& 						& 2(q_yq_z - q_iq_x) \\
						& 						& 1-2(q_x^2 + q_y^2)	
\end{pmatrix},
\end{equation}
which is equivalent to a rotational matrix, that is constructed from euler angles
\begin{equation}
\mat R_m = \begin{pmatrix}
cos(\Pitch)cos(\Yaw)	& cos(\Pitch)sin(\Yaw)	& -sin(\Pitch)			\\
						& 						& sin(\Roll)cos(\Pitch)	\\
						& 						& cos(\Roll)cos(\Pitch)
\end{pmatrix}.
\end{equation}
The euler angles are then
\begin{equation}
\eu e = \begin{pmatrix}\Roll \\ \Pitch \\ \Yaw \end{pmatrix} = 
\begin{pmatrix}
\arctan2(r_{23}, r_{33}) \\
-\arcsin(r_{13}) \\
\arctan2(r_{12}, r_{11})
\end{pmatrix}
\end{equation}
\inHfile{INT32\_EULERS\_OF\_QUAT(e, q)}{pprz\_algebra\_int}
\inHfile{FLOAT\_EULERS\_OF\_QUAT(e, q)}{pprz\_algebra\_float}
\inHfile{DOUBLE\_EULERS\_OF\_QUAT(e, q)}{pprz\_algebra\_float}


\subsubsection*{euler angles derivative from rates}
The transformation from euler angles derivative to rates can be written as a matrix multiplication
\begin{equation}
\begin{pmatrix} p\\q\\r \end{pmatrix} = 
\begin{pmatrix}
- \sin (\Roll) \dot{\Yaw} + \dot{\Roll} \\
\sin (\Roll) \cos (\Pitch) \dot{\Yaw} + \cos (\Roll) \dot{\Pitch} \\
\cos (\Roll) \cos (\Pitch) \dot{\Yaw} - \sin (\Roll) \dot{\Pitch}
\end{pmatrix} \Leftrightarrow \begin{pmatrix} p\\q\\r \end{pmatrix} = 
\begin{pmatrix}
1 & 0 				& -\sin (\Roll) \\
0 & \cos (\Roll)	& \sin (\Roll) \cos (\Pitch) \\
0 & -\sin (\Roll)	& \cos (\Roll) \cos (\Pitch)
\end{pmatrix} \multiplication \begin{pmatrix}
\dot{\Roll} \\
\dot{\Pitch} \\
\dot{\Yaw}
\end{pmatrix}.
\end{equation}
This can be solved easily to
\begin{equation}
\begin{pmatrix}\dot{\Roll} \\ \dot{\Pitch} \\ \dot{\Yaw} \end{pmatrix} = 
\begin{pmatrix}
1 & \frac{ \sin^2 \Roll }{\cos \Pitch}	& \frac{\sin \Roll \cos \Roll}{\cos \Pitch}	\\
0 & \cos \Roll							& -\sin \Roll	\\
0 & \frac{\sin \Roll}{\cos \Pitch}		& \frac{\cos \Roll}{\cos \Pitch}
\end{pmatrix} \multiplication \begin{pmatrix} p\\q\\r \end{pmatrix}.
\end{equation}
Please note the singularity at the \emph{gimbal lock} ($\Pitch = \pm 90^{\circ }$)!
\inHfile{INT32\_EULERS\_DOT\_OF\_RATES(ed, e, r)}{pprz\_algebra\_int}
\inHfile{INT32\_EULERS\_DOT\_321\_OF\_RATES(ed, e, r)}{pprz\_algebra\_int}




\subsection{Other}
\subsubsection*{$-\pi \leq \alpha \leq \pi$ Normalizing}
You have either the option to normalize a single angle to a value between
\begin{equation}
-\pi \leq \alpha \leq \pi
\end{equation}
\inHfile{INT32\_ANGLE\_NORMALIZE(a)}{pprz\_algebra\_int}
\inHfile{FLOAT\_ANGLE\_NORMALIZE(a)}{pprz\_algebra\_float}
or between 
\begin{equation}
0 \leq \alpha \leq 2\pi
\end{equation}
\inHfile{INT32\_COURSE\_NORMALIZE(a)}{pprz\_algebra\_int}

\subsubsection*{$\norm{\eu{e}} $ Norm}
Calculates the 2-norm
\begin{equation}
\norm{\norm{\eu{e}}}_2 = \sqrt{\Roll^2+\Pitch^2+\Yaw^2}
\end{equation}
\inHfile{FLOAT\_EULERS\_NORM(e)}{pprz\_algebra\_float}

\subsubsection*{$min \leq \eu v \leq max$ Bounding}
Bounds the euler angles so that every angle $\Roll$, $\Pitch$ and $\Yaw$ is between \textit{min} and \textit{max}.
\begin{equation}
\eu v \in \mathbb{I}^3, \qquad \mathbb{I} = [min; max]
\end{equation}
\textbf{WARNING:}\\
The function  ``\texttt{EULERS\_BOUND\_CUBE}'' works different than the function \texttt{VECT3\_BOUND\_CUBE} in the case of $min > max$. Here, the lower border \textit{min} has a higher priority than the upper border \textit{max}. So, if $ min > max$ and a value of $ \vect e $ is between those, the value is set to min. \\
\inHfile{EULERS\_BOUND\_CUBE(v, min, max)}{pprz\_algebra}
\mynote{Better naming suggestion: choose e instead of v}
\mynote{The difference between EULERS\_BOUND\_CUBE and VECT3\_BOUND\_CUBE is not very good}
\mynote{No BOUND\_BOX ?}